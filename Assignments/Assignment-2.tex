% Options for packages loaded elsewhere
\PassOptionsToPackage{unicode}{hyperref}
\PassOptionsToPackage{hyphens}{url}
\PassOptionsToPackage{dvipsnames,svgnames,x11names}{xcolor}
%
\documentclass[
]{article}
\usepackage{amsmath,amssymb}
\usepackage{iftex}
\ifPDFTeX
  \usepackage[T1]{fontenc}
  \usepackage[utf8]{inputenc}
  \usepackage{textcomp} % provide euro and other symbols
\else % if luatex or xetex
  \usepackage{unicode-math} % this also loads fontspec
  \defaultfontfeatures{Scale=MatchLowercase}
  \defaultfontfeatures[\rmfamily]{Ligatures=TeX,Scale=1}
\fi
\usepackage{lmodern}
\ifPDFTeX\else
  % xetex/luatex font selection
\fi
% Use upquote if available, for straight quotes in verbatim environments
\IfFileExists{upquote.sty}{\usepackage{upquote}}{}
\IfFileExists{microtype.sty}{% use microtype if available
  \usepackage[]{microtype}
  \UseMicrotypeSet[protrusion]{basicmath} % disable protrusion for tt fonts
}{}
\makeatletter
\@ifundefined{KOMAClassName}{% if non-KOMA class
  \IfFileExists{parskip.sty}{%
    \usepackage{parskip}
  }{% else
    \setlength{\parindent}{0pt}
    \setlength{\parskip}{6pt plus 2pt minus 1pt}}
}{% if KOMA class
  \KOMAoptions{parskip=half}}
\makeatother
\usepackage{xcolor}
\usepackage[margin=1in]{geometry}
\usepackage{graphicx}
\makeatletter
\def\maxwidth{\ifdim\Gin@nat@width>\linewidth\linewidth\else\Gin@nat@width\fi}
\def\maxheight{\ifdim\Gin@nat@height>\textheight\textheight\else\Gin@nat@height\fi}
\makeatother
% Scale images if necessary, so that they will not overflow the page
% margins by default, and it is still possible to overwrite the defaults
% using explicit options in \includegraphics[width, height, ...]{}
\setkeys{Gin}{width=\maxwidth,height=\maxheight,keepaspectratio}
% Set default figure placement to htbp
\makeatletter
\def\fps@figure{htbp}
\makeatother
\setlength{\emergencystretch}{3em} % prevent overfull lines
\providecommand{\tightlist}{%
  \setlength{\itemsep}{0pt}\setlength{\parskip}{0pt}}
\setcounter{secnumdepth}{-\maxdimen} % remove section numbering
\usepackage{booktabs}
\usepackage{xcolor}
\ifLuaTeX
  \usepackage{selnolig}  % disable illegal ligatures
\fi
\usepackage[]{natbib}
\bibliographystyle{plainnat}
\IfFileExists{bookmark.sty}{\usepackage{bookmark}}{\usepackage{hyperref}}
\IfFileExists{xurl.sty}{\usepackage{xurl}}{} % add URL line breaks if available
\urlstyle{same}
\hypersetup{
  pdftitle={Assignment 2},
  pdfauthor={Text as Data},
  colorlinks=true,
  linkcolor={Maroon},
  filecolor={Maroon},
  citecolor={Blue},
  urlcolor={blue},
  pdfcreator={LaTeX via pandoc}}

\title{Assignment 2}
\author{Text as Data}
\date{2023-09-25}

\begin{document}
\maketitle

\hypertarget{parsing-xml-text-data}{%
\subsection{Parsing XML text data}\label{parsing-xml-text-data}}

In this assignment we will access and work with German Parliamentary
data, which is available in XML format
\href{https://www.bundestag.de/services/opendata}{here} (scroll down)
for the last two parliamentary periods. Remember XML format is very like
HTML format, and we can parse it using a scraper and CSS selectors.
Speeches are contained in \texttt{\textless{}rede\textgreater{}}
elements, which each contain a paragraph element describing the speaker,
and paragraph elements recording what they said.

\hypertarget{section}{%
\subsubsection{1.1}\label{section}}

Choose one of the sessions, and retrieve it using R or Python.

\hypertarget{section-1}{%
\subsubsection{1.2}\label{section-1}}

Using a scraper, get a list of all the elements.

\hypertarget{section-2}{%
\subsubsection{1.3}\label{section-2}}

For each element, get the name of the speaker, and a single string
containing everything that they said. Put this into a dataframe.

\hypertarget{section-3}{%
\subsubsection{2.1}\label{section-3}}

Choose a politician, and print the number of speeches they made in this
session

\hypertarget{section-4}{%
\subsubsection{2.2}\label{section-4}}

Print the content of the first speech by the politician you choose.

\hypertarget{section-5}{%
\subsubsection{2.3}\label{section-5}}

Process the list of speeches into a TFIDF matrix. What are the highest
scoring terms in this matrix for the first speech by the politician you
have chosen?

\hypertarget{section-6}{%
\subsubsection{2.4}\label{section-6}}

Using the resource ``Stammdaten aller Abgeordneten seit 1949 im
XML-Format'', retrieve the records pertaining to your chosen politician
and print the information they contain.

  \bibliography{../presentation-resources/MyLibrary.bib}

\end{document}
